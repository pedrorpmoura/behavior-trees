\documentclass[a4paper,UKenglish,cleveref, autoref, thm-restate]{oasics-v2019}
%This is a template for producing OASIcs articles. 
%See oasics-manual.pdf for further information.
%for A4 paper format use option "a4paper", for US-letter use option "letterpaper"
%for british hyphenation rules use option "UKenglish", for american hyphenation rules use option "USenglish"
%for section-numbered lemmas etc., use "numberwithinsect"
%for enabling cleveref support, use "cleveref"
%for enabling autoref support, use "autoref"
%for anonymousing the authors (e.g. for double-blind review), add "anonymous"
%for enabling thm-restate support, use "thm-restate"

%\graphicspath{{./graphics/}}%helpful if your graphic files are in another directory

\bibliographystyle{plainurl}% the mandatory bibstyle


\def\bht{BhTSL}
\title{\bht, Behavior Trees Specification and Processing} %TODO Please add

\titlerunning{\bht, Behavior Trees} %TODO optional, please use if title is longer than one line

\author{Miguel Oliveira}{Centro ALGORITMI, DI, Universidade do Minho, Portugal}{}{}{}
\author{Pedro Mimoso Silva}{Centro ALGORITMI, DI, Universidade do Minho, Portugal}{}{}{}
\author{Pedro Moura}{Centro ALGORITMI, DI, Universidade do Minho, Portugal}{}{}{}
\author{José João Almeida}{Centro ALGORITMI, DI, Universidade do Minho, Portugal}{}{}{}
\author{Pedro Rangel Henriques}{Centro ALGORITMI, DI, Universidade do Minho, Portugal}{}{}{}


\authorrunning{M. Oliveira et al.} %TODO mandatory. First: Use abbreviated first/middle names. Second (only in severe cases): Use first author plus 'et al.'

\Copyright{John Q. Public and Joan R. Public} %TODO mandatory, please use full first names. OASIcs license is "CC-BY";  http://creativecommons.org/licenses/by/3.0/

\ccsdesc[100]{\textcolor{red}{Replace ccsdesc macro with valid one}} %TODO mandatory: Please choose ACM 2012 classifications from https://dl.acm.org/ccs/ccs_flat.cfm 

\keywords{Game development, Behavior trees, DSL, NPC, Code generation} %TODO mandatory; please add comma-separated list of keywords

\acknowledgements{I want to thank \dots}%optional

%\nolinenumbers %uncomment to disable line numbering

%\hideOASIcs  %uncomment to remove references to OASIcs series (logo, DOI, ...), e.g. when preparing a pre-final version to be uploaded to arXiv or another public repository

%Editor-only macros:: begin (do not touch as author)%%%%%%%%%%%%%%%%%%%%%%%%%%%%%%%%%%
\EventEditors{John Q. Open and Joan R. Access}
\EventNoEds{2}
\EventLongTitle{42nd Conference on Very Important Topics (CVIT 2016)}
\EventShortTitle{CVIT 2016}
\EventAcronym{CVIT}
\EventYear{2016}
\EventDate{December 24--27, 2016}
\EventLocation{Little Whinging, United Kingdom}
\EventLogo{}
\SeriesVolume{42}
\ArticleNo{23}
%%%%%%%%%%%%%%%%%%%%%%%%%%%%%%%%%%%%%%%%%%%%%%%%%%%%%%

\begin{document}

\maketitle

%TODO mandatory: add short abstract of the document
\begin{abstract}
In the context of game development, there is always the need for describing behaviors for various entities, whether NPCs or even the world itself.
That need requires a formalism to describe properly such behaviors.

As gaming industry has been growing, many approaches were proposed.
First, finite state machines were used and evolved to hierarchical state machines.
As this wasn't enough, a more powerful concept appeared.
Instead of using states for describing behaviors, people started to use tasks.
This concept was incorporated in behavior trees.

This paper focuses in the specification and processing of these behavior trees.
A DSL designed for that purpose will be introduced.
It will also be discussed a generator that produces \LaTeX\ diagrams to document the trees, and a Python module to implement the behavior described.
Aditionally, a simulator will be presented. 
These achievements will be illustrated using a concrete game as a case study.
\end{abstract}


\section{Introduction}
\label{sec:introduction}

At some point in the video-game history, NPCs were introduced. 
With them came the need to describe behaviors.
And with this behaviors came the need of the existence of a formalism so that they can be properly specified.

As time passed by, various approaches were proposed and used, like finite and hierarchical state machines.
These are state-based behaviors, that is, the behaviors are described through states.
Altough this is a clear and simplistic way to represent and visualize small behaviors, it becomes unsustainable when dealing with bigger and more complex behaviors.
Some time later, a new and more powerful concept was introduced: using tasks instead of states to describe behaviors.
This concept is incorporated in what we call behavior trees.

Behavior trees were first used in the videogame industry in the development of the game \textit{Halo 2}, released in 2004.
The idea is that people create a complex behavior by only programming actions (or tasks) and then design a tree structure whose leaf nodes are actions and the inner nodes determine the NPC's decision making.
Not only these provide an easy and intuitive way of visualizing and designing behaviors, they also provide a good way to work with scalability through modularity, solving the biggest issue from state-based design.
Since then, multiple gaming companies adopted this concept and, in recent years, behavior trees are also being used in areas like Artificial Inteligence and Robotics.

This paper focuses in the specification and processing of these behavior trees.
For this purpose, it will be introduced an DSL, and a compiler that can generate a \LaTeX\ diagram to visualize and a Python module to implement the described behavior.

These achievements will be illustrated using a concrete game as a case study.



- Motivation
- Design goals, ir direto ao assunto

\section{State of the Art}
\label{sec:state-of-the-art}

In this area there is some interesting projects that utilize Behavior trees as the main focus to describe their NPCs. 
In the gaming industry there are several engines that are utilized to produce games, this engines saw the potencial of incorporating Behavior Trees as a method of describing the behavior of NPCs.
Two major engines that use Behavior Trees are Unreal Engine and Unity, in their case they chose to go user friendly and utilize graphical design to represent the trees.

To better understand how does a behavior tree work, its composed of control flow nodes and leafs named execution nodes.
Each node linked its called parent and child. 
The root node does not have a parent the other nodes have exatly one parent.

-FICAMOS AQUI

- conceitos básicos

\section{Architecture and Specification}
\label{sec:arc-spec}

- Gramática
- Especificação


\section{Tools}
\label{sec:tools}
- processador

\section{Example}
\label{sec:example}

\section{Conclusion}
\label{sec:conclusion}

\end{document}
