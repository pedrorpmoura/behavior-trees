
\documentclass[a4paper,UKenglish,cleveref, autoref, thm-restate]{oasics-v2019}
%This is a template for producing OASIcs articles. 
%See oasics-manual.pdf for further information.
%for A4 paper format use option "a4paper", for US-letter use option "letterpaper"
%for british hyphenation rules use option "UKenglish", for american hyphenation rules use option "USenglish"
%for section-numbered lemmas etc., use "numberwithinsect"
%for enabling cleveref support, use "cleveref"
%for enabling autoref support, use "autoref"
%for anonymousing the authors (e.g. for double-blind review), add "anonymous"
%for enabling thm-restate support, use "thm-restate"

%\graphicspath{{./graphics/}}%helpful if your graphic files are in another directory

\bibliographystyle{plainurl}% the mandatory bibstyle


\def\bht{BhTSL}
\title{\bht, Behavior Trees Specification and Processing} %TODO Please add

\titlerunning{\bht, Behavior Trees} %TODO optional, please use if title is longer than one line

\author{Miguel Oliveira}{Centro ALGORITMI, DI, Universidade do Minho, Portugal}{}{}{}
\author{Pedro Mimoso Silva}{Centro ALGORITMI, DI, Universidade do Minho, Portugal}{}{}{}
\author{Pedro Moura}{Centro ALGORITMI, DI, Universidade do Minho, Portugal}{}{}{}
\author{José João Almeida}{Centro ALGORITMI, DI, Universidade do Minho, Portugal}{}{}{}
\author{Pedro Rangel Henriques}{Centro ALGORITMI, DI, Universidade do Minho, Portugal}{}{}{}


\authorrunning{M. Oliveira et al.} %TODO mandatory. First: Use abbreviated first/middle names. Second (only in severe cases): Use first author plus 'et al.'

\Copyright{John Q. Public and Joan R. Public} %TODO mandatory, please use full first names. OASIcs license is "CC-BY";  http://creativecommons.org/licenses/by/3.0/

\ccsdesc[100]{\textcolor{red}{Replace ccsdesc macro with valid one}} %TODO mandatory: Please choose ACM 2012 classifications from https://dl.acm.org/ccs/ccs_flat.cfm 

\keywords{Game development, Behavior trees, DSL, NPC, Code generation} %TODO mandatory; please add comma-separated list of keywords

\acknowledgements{I want to thank \dots}%optional

%\nolinenumbers %uncomment to disable line numbering

%\hideOASIcs  %uncomment to remove references to OASIcs series (logo, DOI, ...), e.g. when preparing a pre-final version to be uploaded to arXiv or another public repository

%Editor-only macros:: begin (do not touch as author)%%%%%%%%%%%%%%%%%%%%%%%%%%%%%%%%%%
\EventEditors{John Q. Open and Joan R. Access}
\EventNoEds{2}
\EventLongTitle{42nd Conference on Very Important Topics (CVIT 2016)}
\EventShortTitle{CVIT 2016}
\EventAcronym{CVIT}
\EventYear{2016}
\EventDate{December 24--27, 2016}
\EventLocation{Little Whinging, United Kingdom}
\EventLogo{}
\SeriesVolume{42}
\ArticleNo{23}
%%%%%%%%%%%%%%%%%%%%%%%%%%%%%%%%%%%%%%%%%%%%%%%%%%%%%%

\begin{document}

\maketitle

%TODO mandatory: add short abstract of the document
\begin{abstract}
In the context of game development, there is always the need for describing behaviors for various entities, whether NPCs or even the world itself.
That need requires a formalism to describe properly such behaviors.

As gaming industry has been growing, many approaches were proposed.
First, finite state machines were used and evolved to hierarchical state machines.
As this wasn't enough, a more powerful concept appeared.
Instead of using states for describing behaviors, people started to use tasks.
This concept was incorporated in behavior trees.

This paper focuses in the specification and processing of these behavior trees.
A DSL designed for that purpose will be introduced.
It will also be discussed a generator that produces \LaTeX\ diagrams to document the trees, and a Python module to implement the behavior described.
Aditionally, a simulator will be presented. 
These achievements will be illustrated using a concrete game as a case study.
\end{abstract}


\section{Introduction}
\label{sec:introduction}

In some point of the video-game history, NPCs were introduced. 
With them came the need to describe behaviors.
And with this behaviors came the need of the existence of a formalismm so that they can be properly specified.


As time passed by, various approaches were proposed and used. Finite state machines, hierarchical state machines and decision trees are some examples. 
However some companies decided that it was easy to humans to understand behaviors instead of states, so originated behavior trees used to model NPCS.
It was first used in games like Halo, Bioshock, and Spore, and then other gaming companies saw the potencial, and now they are being used in game environments such as PyGame, Unreal Engine and Unity that are a majority of the platforms used to create games.
In recent days it expanded to be used in Artificial Inteligence and in Robotics



-----------------------------



Behavior trees became popular for their development paradigm: being able to create a complex behavior by only programming the NPC's actions and then designing a tree structure (usually through drag and drop) whose leaf nodes are actions and whose inner nodes determine the NPC's decision making. Behavior trees are visually intuitive and easy to design, test, and debug, and provide more modularity, scalability, and reusability than other behavior creation methods.

Over the years, the diverse implementations of behavior trees kept improving both in efficiency and capabilities to satisfy the demands of the industry, until they evolved into event-driven behavior trees[15]. Event-driven behavior trees solved some scalability issues of classical behavior trees by changing how the tree internally handles its execution, and by introducing a new type of node that can react to events and abort running nodes. Nowadays, the concept of event-driven behavior tree is a standard and used in most of the implementations, even though they are still called "behavior trees" for simplicity.

- Motivation
- Design goals, ir direto ao assunto

\section{State of the Art}
\label{sec:state-of-the-art}
- conceitos básicos

\section{Architecture and Specification}
\label{sec:arc-spec}

- Gramática
- Especificação


\section{Tools}
\label{sec:tools}
- processador

\section{Example}
\label{sec:example}

\section{Conclusion}
\label{sec:conclusion}

\end{document}
