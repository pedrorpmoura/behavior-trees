% Chapter Template

\chapter{Linguagem} % Main chapter title

\label{Chapter3} % Change X to a consecutive number; for referencing this chapter elsewhere, use \ref{ChapterX}

Tal como foi referido anteriormente, o objetivo deste projeto é o desenvolvimento de uma linguagem de modo a representar \textit{Behavior Trees}.

Agora que já explicamos o que são estas árvores, podemos avançar para a especificação da linguagem.

\section{Estrutura}
Chamaremos a cada ficheiro da nossa linguagem um ficheiro de comportamento.

Um ficheiro de comportamento é composto por 3 partes: \textit{behavior}, \textit{definitions} e \textit{code} (opcional).

\subsection{\textit{Behavior}}
O \textit{behavior} é a árvore principal do programa, e é esta que é utilizada para especificar o comportamento representado pelo ficheiro.
De certa forma, pode ser visto como a função \textit{main} de um programa.

\subsection{\textit{Definitions}}


\subsection{\textit{Code}}


\section{Sintaxe}


%Na linguagem, pode ser descrito como se segue:
%
%\begin{lstlisting} 
%    behavior : [
%        tree
%    ]
%\end{lstlisting}
%
%Dentro do \textit{behavior}, começaremos a criar a nossa árvore de comportamento.
%Seguindo a estrutura do capítulo anterior, os nodos podem ser especificados das seguintes formas:
%\begin{figure}[H]
%\centering
%
%\begin{subfigure}{0.45\linewidth}
%\begin{lstlisting}
%    sequence : [
%        child1,
%        child2,
%        ...
%        childn
%    ]
%\end{lstlisting}
%\caption{}
%\end{subfigure}
%\hfil
%\begin{subfigure}{0.45\linewidth}
%\begin{lstlisting}
%    selector : [
%        child1,
%        child2,
%        ...
%        childn
%    ]
%\end{lstlisting}
%\caption{}
%\end{subfigure}
%\begin{subfigure}{0.45\linewidth}
%\begin{lstlisting}
%    prob_selector : [
%        P1 -> child1,
%        P2 -> child2,
%        ...
%        Pn -> childn
%    ]
%\end{lstlisting}
%\caption{}
%\end{subfigure}
%\hfil
%\begin{subfigure}{0.45\linewidth}
%\begin{lstlisting}
%    parallel : M [
%        child1,
%        child2,
%        ...
%        childn
%    ]
%\end{lstlisting}
%\caption{}
%\end{subfigure}
%\begin{subfigure}{0.45\linewidth}
%\begin{lstlisting}
%    decorator : POLICY [
%        child
%    ]
%\end{lstlisting}
%\caption{}
%\end{subfigure}
%\hfil
%\begin{subfigure}{0.45\linewidth}
%\begin{lstlisting}
%    condition : $CONDITION
%
%    action : $ACTION
%\end{lstlisting}
%\caption{}
%\end{subfigure}
%\end{figure}