% Chapter Template

\chapter{\textit{Behavior Trees} - Básicos} % Main chapter title

\label{Chapter2} % Change X to a consecutive number; for referencing this chapter elsewhere, use \ref{ChapterX}



\section{O que são \textit{Behavior Trees}?}
\textit{Behavior Trees}, ou BT, são estruturas que controlam a execução de um conjunto ações por parte de um agente autónomo, de forma a descreverem o seu comportamento.
Este tipo de estrutura começou por ser utilizado especialmente em videojogos (para simular o comportamento de NPCs), mas com o passar do tempo áreas como a Robótica ou a Inteligência Artificial também o começaram a usar, devido à sua capacidade modular e reativa.

Formalmente, uma BT é uma árvore com raiz onde os nodos internos são chamados \textit{control flow nodes} e as folhas são chamadas de \textit{execution nodes}. Para cada nodo conetado usamos a terminologia de \textit{parent} (pai) e \textit{child} (filho).
A raiz é o único nodo que não tem pais, todos os outros nodos têm exatamente um pai.
Os \textit{control flow nodes} têm pelo menos um filho, já os \textit{execution nodes} não têm nenhum.

A figura \ref{fig:2.1} mostra um exemplo de uma BT, onde o nodo a verde é a raiz, os nodos a cinzento são \textit{control flow nodes} e os restantes são \textit{execution nodes}.

\begin{figure}[H]
\centering
\begin{behavior}
    [\rootnode
        [\sequence
            [\action{Walk to door}]
            [\selector
                [\action{Open door}]
                [\sequence
                    [\action{Unlock door}]
                    [\action{Open door}]
                ]
                [\action{Smash door}]
            ]
            [\action{Walk through door}]
            [\action{Close door}]
        ]
    ]
\end{behavior}
\caption{Exemplo de uma BT.}
\label{fig:2.1}
\end{figure}

\section{Execução}
Uma BT começa a sua execução pelo nodo raiz que gera sinais chamados \textit{ticks} com uma determinada frequência, que são enviados para os seus filhos.
Estes sinais permitem a execução dos nodos.

Qualquer nodo, não importa o tipo, ao ser executado retorna um de três estados: 
\begin{itemize}
    \item \textit{Success} - foi executado com sucesso;
    \item \textit{Failure} - não conseguiu ser executado;
    \item \textit{Running} - ainda está a executar.
\end{itemize}




\section{Nodos}

\subsection{\textit{Control Flow Nodes} - Clássicos}
\textit{Control flow nodes} são nodos estruturais, que não têm qualquer impacto no estado global do sistema.
Na formulação clássica, existem 4 categorias deste tipo de nodo: \textit{Sequence}, \textit{Selector (ou Fallback)}, \textit{Parallel} e \textit{Decorator}.

\paragraph{Sequence}{
    Um nodo \textit{Sequence} visita (envia \textit{ticks}) todos os filhos por ordem, começando pelo primeiro, e se este retornar \textit{Success}, chama o segundo, e por aí em diante.
    Caso um filho falhe, o nodo \textit{Sequence} retorna \textit{Failure} imediatamente.
    Caso todos os filhos sucedam, o nodo retorna \textit{Success}.
    Caso um filho retorne \textit{Running}, o nodo retorna também \textit{Running}.

    \begin{figure}[H]
    \centering
    \begin{behavior}
        [\sequence
            [\action{Child 1}]
            [\action{Child 2}]
            [{\textbf{. . .}}, inner sep=10pt]
            [\action{Child N}]
        ]
    \end{behavior}
    \caption{Estrutura de um nodo \textit{Sequence}.}
    \label{fig:2.2}
    \end{figure}
}

\paragraph{Selector}{
    Tal como o \textit{Sequence}, o nodo \textit{Selector} visita todos os filhos por ordem, mas só avança para o próximo se o filho que está a ser executado retorne \textit{Failure}.
    Caso um filho suceda, o nodo \textit{Sequence} retorna \textit{Sucess} imediatamente.
    Caso todos os filhos falhem, retorna \textit{Failure}.
    Caso um filho retorne \textit{Running}, o nodo retorna também \textit{Running}.

    \begin{figure}[H]
    \centering
    \begin{behavior}
        [\selector
            [\action{Child 1}]
            [\action{Child 2}]
            [{\textbf{. . .}}, inner sep=10pt]
            [\action{Child N}]
        ]
    \end{behavior}
    \caption{Estrutura de um nodo \textit{Selector}.}
    \label{fig:2.3}
    \end{figure}
}

\paragraph{Parallel}{
    Um nodo \textit{Parallel}, como o nome indica, visita todos os filhos paralelamente.
    Para $M \leq N$, retorna \textit{Success} caso $M$ filhos sucedam, e retorna \textit{Failure} caso $N - M + 1$ filhos retornem \textit{Failure}.
    Caso contrário, retorna \textit{Running}.

    \begin{figure}[H]
    \centering
    \begin{behavior}
        [\parallel{M}
            [\action{Child 1}]
            [\action{Child 2}]
            [{\textbf{. . .}}, inner sep=10pt]
            [\action{Child N}]
        ]
    \end{behavior}
    \caption{Estrutura de um nodo \textit{Parallel} com taxa de sucesso $M$.}
    \label{fig:2.4}
    \end{figure}
}

\paragraph{Decorator}
POR FAZER


\subsection{\textit{Control Flow Nodes} - Novos}
POR FAZER


\subsection{\textit{Execution Nodes}}
\textit{Execution nodes} são os nodos mais simples, porém os mais poderosos, pois contêm os testes ou ações que serão implementados pelo sistema.
Existem 2 categorias para este tipo de nodos: \textit{Action} e \textit{Condition}.

\paragraph{Action}{
    Quando recebe \textit{ticks}, um nodo \textit{Action} executa um ou mais comandos.
    Retorna \textit{Success} se a ação foi corretamente completada ou \textit{Failure} se a ação falhou.
    Enquanto está a ser executado, retorna \textit{Running}.

    \begin{figure}[H]
    \centering
    \begin{behavior}
        [\action{Action}]
    \end{behavior}
    \caption{Estrutura de um nodo \textit{Action}.}
    \label{fig:2.5}
    \end{figure}
}

\paragraph{Condition}{
    Quando recebe \textit{ticks}, um nodo \textit{Condition} verifica uma proposição.
    Retorna \textit{Success} ou \textit{Failure} dependendo se a proposição é válida ou não.
    De notar que um nodo \textit{Condition} nunca retorna um estado \textit{Running}.

    \begin{figure}[H]
    \centering
    \begin{behavior}
        [\condition{Condition}]
    \end{behavior}
    \caption{Estrutura de um nodo \textit{Condition}.}
    \label{fig:2.6}
    \end{figure}

    \hr
    \paragraph{\textit{NOTA:}}{
        \textit{
        Quando juntamos o nodo Condition com os nodos Sequence e Selector podemos criar uma expressão condicional (if-then-else). Vejamos o seguinte exemplo:
        }
        \begin{figure}[H]
        \centering
        \begin{behavior}
            [\selector
                [\sequence
                    [\condition{Shop is open}]
                    [\action{Go shopping}]
                ]
                [\action{Go home}]
            ]
        \end{behavior}
        \caption{Exemplo de uma expressão condicional numa BT.}
        \label{fig:2.7}
        \end{figure}

        \textit{
        Como podemos ver, a escolha da ação a executar alterna entre ir às compras ou ir embora consoante a loja esteja aberta ou não. Ou seja,
        }

        \begin{center}
            \textit{
            \underline{\textbf{if}} Shop is open \underline{\textbf{then}} Go shopping \underline{\textbf{else}} Go home
            }
        \end{center}
    }
    \hr
}


\section{Exemplo - \textit{PACMAN}}


