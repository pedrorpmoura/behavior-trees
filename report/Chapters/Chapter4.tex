% Chapter Template

\chapter{Compilador} % Main chapter title

\label{Chapter4} % Change X to a consecutive number; for referencing this chapter elsewhere, use \ref{ChapterX}

\section{Análise léxica \textit{(tokenization)}}
A primeira etapa no desenvolvimento de um compilador consiste na criação de um analisador léxico, ou \textit{lexer}, que é responsável por converter uma sequência de caracteres numa sequência de \textit{tokens}.
Um \textit{token} é uma \textit{string} com um significado atribuído, e pode ser estruturado como um par que contém um nome e um valor opcional.

A seguinte tabela mostra os \textit{tokens} utilizados para esta análise:


\begin{center}
    \begin{tabular}{ | p {5cm} | p {5cm} | } 
        \hline
            \multicolumn{2}{ |c| }{\textit{\textbf{Tokens}}} \\
        \hline
            \multicolumn{1}{ |c| }{\textbf{Nome}} & \multicolumn{1}{ |c| }{\textbf{Valor}} \\
        \hline
        \hline
            literais            & \texttt{({[]}),:\%}           \\ \hline
            t\_RIGHTARROW       & \texttt{->}                   \\ \hline
            t\_BEHAVIOR         & \texttt{\textbackslash bbehavior\textbackslash b}     \\ \hline
            t\_SEQUENCE         & \texttt{\textbackslash bsequence\textbackslash b}     \\ \hline
            t\_SELECTOR         & \texttt{\textbackslash bselector\textbackslash b}     \\ \hline
            t\_PROBSELECTOR     & \texttt{\textbackslash bprobselector\textbackslash b} \\ \hline
            t\_PARALLEL         & \texttt{\textbackslash bparallel\textbackslash b}     \\ \hline
            t\_DECORATOR        & \texttt{\textbackslash bdecorator\textbackslash b}    \\ \hline
            t\_CONDITION        & \texttt{\textbackslash bcondition\textbackslash b}    \\ \hline
            t\_ACTION           & \texttt{\textbackslash baction\textbackslash b}       \\ \hline
            t\_INVERTER         & \texttt{\textbackslash bINVERTER\textbackslash b}     \\ \hline
            t\_MAXTRIES         & \texttt{\textbackslash bMAXTRIES\textbackslash b}     \\ \hline
            t\_MAXSECONDS       & \texttt{\textbackslash bMAXSECONDS\textbackslash b}   \\ \hline
            t\_INT              & \texttt{\textbackslash d+}                            \\ \hline
            t\_VAR              & \texttt{\$\textbackslash w+}                          \\ \hline
            t\_NODENAME         & \texttt{\textbackslash b\textbackslash w+\textbackslash b}   \\ \hline
            t\_CODE             & \texttt{\%\%(.|\textbackslash n)+} \\
            
        \hline
    \end{tabular}
\end{center}

\section{Análise sintática \textit{(parsing)}}



\section{Análise semântica}
