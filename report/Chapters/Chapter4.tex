% Chapter Template

\chapter{Compilador} % Main chapter title

\label{Chapter4} % Change X to a consecutive number; for referencing this chapter elsewhere, use \ref{ChapterX}

\section{Análise léxica \textit{(tokenization)}}
A primeira etapa no desenvolvimento de um compilador consiste na criação de um analisador léxico, ou \textit{lexer}, que é responsável por converter uma sequência de caracteres numa sequência de \textit{tokens}.
Um \textit{token} é uma \textit{string} com um significado atribuído, e pode ser estruturado como um par que contém um nome e um valor opcional.

A seguinte tabela de valores mostra os \textit{tokens} utilizados para esta análise:

\begin{table}[H]
    \centering
    \begin{tabular}{ | p {5cm} | p {5cm} | } 
        \hline
            \multicolumn{2}{ |c| }{\textit{\textbf{Tokens}}} \\
        \hline
            \multicolumn{1}{ |c| }{\textbf{Nome}} & \multicolumn{1}{ |c| }{\textbf{Valor}} \\
        \hline
        \hline
            literais            & \texttt{({[]}),:\%}           \\ \hline
            RIGHTARROW       & \texttt{->}                   \\ \hline
            BEHAVIOR         & \texttt{\textbackslash bbehavior\textbackslash b}     \\ \hline
            SEQUENCE         & \texttt{\textbackslash bsequence\textbackslash b}     \\ \hline
            SELECTOR         & \texttt{\textbackslash bselector\textbackslash b}     \\ \hline
            PROBSELECTOR     & \texttt{\textbackslash bprobselector\textbackslash b} \\ \hline
            PARALLEL         & \texttt{\textbackslash bparallel\textbackslash b}     \\ \hline
            DECORATOR        & \texttt{\textbackslash bdecorator\textbackslash b}    \\ \hline
            CONDITION        & \texttt{\textbackslash bcondition\textbackslash b}    \\ \hline
            ACTION           & \texttt{\textbackslash baction\textbackslash b}       \\ \hline
            INVERTER         & \texttt{\textbackslash bINVERTER\textbackslash b}     \\ \hline
            MAXTRIES         & \texttt{\textbackslash bMAXTRIES\textbackslash b}     \\ \hline
            MAXSECONDS       & \texttt{\textbackslash bMAXSECONDS\textbackslash b}   \\ \hline
            INT              & \texttt{\textbackslash d+}                            \\ \hline
            VAR              & \texttt{\$\textbackslash w+}                          \\ \hline
            NODENAME         & \texttt{\textbackslash b\textbackslash w+\textbackslash b}   \\ \hline
            CODE             & \texttt{\%\%(.|\textbackslash n)+} \\
        \hline
    \end{tabular}
    \caption{Tabela de valores do \textit{lexer}.}
    \label{tab:4.1}
\end{table}


\section{Análise sintática \textit{(parsing)}}
A análise sintática é o processo de analizar uma sequência de símbolos consoante uma gramática.
De uma forma muito simples, uma gramática é um conjunto de regras que nos diz se uma determinada \textit{string} pertence a uma certa linguagem ou se está gramaticalmente incorreta.

Os símbolos podem ser de dois tipos: \textit{terminais} ou \textit{não terminais} (\textit{tokens}).
Os símbolos terminais, ou \textit{tokens}, são os símbolos elementares da linguagem.
Os não terminais são substituídos pelos terminais de acordo com as regras de produção da gramática.

Para a nossa linguagem, temos a seguinte gramática, onde os símbolos em minúsculo são os não terminais e os que estão em maiúsculo são os terminais:

\begin{lstlisting}
    root : behavior CODE
         | behavior definitions CODE
         | definition behavior CODE

    behavior : BEHAVIOR ':' '[' node ']'

    node : SEQUENCE ':' '[' nodes ']'
         | SEQUENCE ':' VAR
         | MEMORY SEQUENCE ':' '[' nodes ']'
         | MEMORY SEQUENCE ':' VAR
         | SELECTOR ':' '[' nodes ']'
         | SELECTOR ':' VAR
         | MEMORY SELECTOR ':' '[' nodes ']'
         | MEMORY SELECTOR ':' VAR
         | PROBSELECTOR ':' '[' prob_nodes ']'
         | PROBSELECTOR ':' VAR
         | MEMORY PROBSELECTOR ':' '[' prob_nodes ']'
         | MEMORY PROBSELECTOR ':' VAR
         | PARALLEL ':' INT '[' nodes ']'
         | PARALLEL ':' VAR
         | DECORATOR ':' INVERTER '[' node ']'
         | DECORATOR ':' MAXTRIES '(' INT ')' '[' node ']'
         | DECORATOR ':' MAXSECONDS '(' INT ')' '[' node ']'
         | DECORATOR ':' VAR
         | CONDITION ':' VAR
         | ACTION ':' VAR
    
    nodes : nodes ',' node
          | node
    
    prob_nodes : prob_nodes ',' prob_node
               | prob_node
    
    prob_node : VAR RIGHTARROW node

    definitions : definitions definition
               | definition

    definition : SEQUENCE NODENAME ':' '[' nodes ']'
              | SELECTOR NODENAME ':' '[' nodes ']'
              | PROBSELECTOR NODENAME ':' '[' prob_nodes ']'
              | PARALLEL NODENAME ':' INT '[' nodes ']'
              | DECORATOR NODENAME ':' INVERTER '[' node ']'
              | DECORATOR NODENAME ':' MAXTRIES '(' INT ')' '[' node ']'
              | DECORATOR NODENAME ':' MAXSECONDS '(' INT ')' '[' node ']'
\end{lstlisting}


\section{Análise semântica}
